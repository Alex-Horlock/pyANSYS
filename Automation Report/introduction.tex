%the context of the project; previous work; need for improvements.


There is no product available aimed at reducing the drag of intermodal containers during road transit. Due to their bluff body nature, increasing the base pressure is an effective method of reducing drag. Therefore this project focusses on developing a rear attachment device to increase the base pressure on an intermodal container. Computational and experimental studies are undertaken in order to optimise the geometry and configuration of the proposed device. \\

%This project was aimed at providing a solution to this problem; namely a retrofit device compatibile with intermodal containers that has been proven, computationally and experimentally, to reduce drag. Furthermore, it was desired to prove the economic viability of this device, and to provide a mechanically sound design.\\

%There are approximately 32 million intermodal containers in use around the world. The standardised geometry of these was specified in 1961. These containers have little consideration for aerodynamics, despite being transported great distances by road. Given the scale of the container transport industry, a small reduction in aerodynamic drag would lead to a large economic and environmental benefit.\\

